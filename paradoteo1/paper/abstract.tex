The rapid adoption of Large Language Models (LLMs) has intensified concerns about privacy breaches and copyright violations, given these models' capacity to retain sensitive information encountered during training. Mechanistic Unlearning (MU) has emerged as a solution, proposing to remove specific information without full retraining while preserving the model's overall performance. However, many unlearning techniques often fail to truly erase knowledge, merely suppressing it while it remains encoded in the model's internal representations. This project investigates the efficacy of unlearning methodologies through the lens of Mechanistic Interpretability (MI)—a field dedicated to reverse-engineering the specific circuits and features that govern model behavior. My work focuses on small-scale transformer models (e.g. GPT-Nano, GPT-2 Small) to enable inspection of weight updates. The primary objective is to reproduce established unlearning baselines on these tractable architectures and utilize MI tools to distinguish between superficial suppression and genuine knowledge erasure.
